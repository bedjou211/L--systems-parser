\section{Conclusion}
  En conclusion, ce rapport présente le développement d'un interpréteur de L-système en Java pour générer des images 2D/3D de structures végétales à partir de règles de réécriture. Le projet comprend la conception d'un parser, d'un moteur de réécriture et d'un moteur de rendu graphique, ainsi que des expérimentations pour évaluer la performance et la qualité des images générées. Les extensions possibles du projet, telles que la prise en charge des L-systèmes stochastiques et/ou contextuels, sont également discutées. Ce projet offre ainsi une base solide pour la génération de structures végétales réalistes dans des applications de jeux vidéos, d'animation, de modélisation graphique, et peut être étendu pour répondre à des besoins spécifiques.
    Les résultats obtenus dans ce projet de groupe ont démontré la capacité des membres à collaborer, à mettre en œuvre leurs compétences individuelles, et à atteindre les objectifs communs. Les améliorations apportées tout au long du projet, notamment l'ajout d'une animation avec un timer, ont permis d'enrichir l'expérience utilisateur et d'améliorer la qualité du produit final.